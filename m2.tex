\section*{Midterm 2 (Feb 28, 2020)}

\problem{
    f15m2q1
}{
    Give 1 technique that can be used to reduce the Miss Rate.
}{
    \todo
}

\problem{
    f15m2q2
}{
    Give 1 technique that can be used to reduce the Hit Time.
}{
    \todo
}

\problem{
    f14m2q4
}{
    What does ROB stand for, and why is it used in modern advanced pipelines? 
}{
    Reorder Buffer. It supports Precise Interrupt.
}

\problem{
    f16m2q1
}{
    Writes to a cache are inherently slower than reads from a cache - why?
}{
     \todo
}

\problem{
    f16m2q2
}{
    What is the primary difference between Tomasulo’s algorithm and Scoreboarding?
}{
     \todo
}

\problem{
    f16m2q3
}{
    Compilers have trouble optimizing code that involves reads and writes to memory. Why? (The answer has nothing to do with how slow memory is - that is a different problem altogether).
}{
     \todo
}

\problem{
    f16m2q4
}{
    Why is there a desire to create larger basic blocks? Give one example of a way to create a bigger basic block.
}{
     \todo
}

\problem{
    f16m2q5
}{
    Processors have been built that were able to issue 8 instructions at a time using a fast clock. However, these processors are no longer being built - why not? Why would you choose a 3-issue machine over an 8-issue machine, if the clock rates were the same?
}{
     \todo
}

\problem{
    f16m2q6
}{
    Intel uses a Tournament predictor in some of their processors. Describe what it is, and why it is used. Your description can (and probably should) include sketches and drawings.
}{
     \todo
}

\problem{
    f16m2q7
}{
    Why is loop unrolling a valuable optimization technique? What are 2 challenges to using it?
}{
     \todo
}

\problem{
    f16m2q8
}{
    The book states that slow and wide architectures can be more power efficient than fast and nar- row architectures. Explain why. Also, explain the underlying assumption that is being made, and why it is that we are still making narrow fast machines.
}{
     \todo
}

\problem{
    f16m2q9
}{
    Supporting precise interrupts in machines that allow out of order completion is a challenge. Briefly explain what a precise interrupt is, why it is a challenge in OOO machines, and describe the main technique used today to provide precise interrupts.
}{
     \todo
}

\problem{
    f16m2q10
}{
    Briefly outline how a Vector machine works, and what type of parallelism it is exploiting.
}{
     \todo
}

\problem{
    f16m2q11
}{
    What is the primary difference between superscalar and VLIW processors? Give one advantage and one disadvantage to using each approach. (These have to be different - in other words, if the advantage of superscalar is X, then you can’t say a disadvantage of VLIW is that it can’t do X.)
}{
     \todo
}
