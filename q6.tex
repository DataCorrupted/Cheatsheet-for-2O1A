
\problem{
    f16m2q16
}{
    Assuming a 19-bit address and a 256-byte Direct Mapped cache with a linesize=2, show how an address is partitioned/interpreted by the cache.
}{
    offset = 1 bit \\
    256/2 = 128 = $2^7$  entry = 7 bits \\
    tag  = 11 bits
}

\problem{
    f16m2q17
}{
    Assuming an 19-bit address and an 80-byte 10-way Set Associative cache with a linesize=4, show how an address is partitioned/interpreted by the cache.
}{
    offset = 2 bits \\
    80/(10*4) = 2 = $2^1$  set = 1 bit \\
    tag = 16 bits
}

\problem{
    f16m2q18
}{
    Assuming an 19-bit address and a 328-byte Fully Associative cache with a linesize=8, show how an address is partitioned/interpreted by the cache.
}{
    offset = 3 bits \\
    tag = 16 bits
}

\problem{
    f16m2q19
}{
    Given a 1 Megabyte physical memory, a 22-bit Virtual address, and a page size of 1K bytes, write down the number of entries in the Page Table, and the width of each entry.
}{
   1M = $2^{20}$ \\
   1K = $2^{10}$ \\
   $(2^{20})/(2^{10}) = 2^{10} \rightarrow$ 10-bit wide \\
   PM: 10 | 10 \\
   VM: 12 | 10 \\
   $2^{12}$ entries
}

\problem{
    f16m2q20
}{
    Given a 1 Megabyte physical memory, a 34-bit Virtual address, and a page size of 2K bytes, write down the number of entries in the Page Table, and the width of each entry. Is there a problem with this configuration? If so, what is the problem?
}{
    1M = $2^{20}$ \\
    2K = $2^{11}$ \\
    $(2^{20})/(2^{11}) = 2^9 \rightarrow$ 9-bit wide \\
    PM: 9 | 11 \\
    VM: 23 | 11 \\
    $2^{23}$ entries \\
    Problem: It is too large to store in 1M physical memory. We can use two page tables. One is base, another one is secondary.
}