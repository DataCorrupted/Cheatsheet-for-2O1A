\section*{Week 3 (1/20 - 1/26; Appx.A, Quiz 2)}

\problem{
    w19f0q17
}{
    If I add processors but keep the job size the same, am I measuring strong or weak scaling? 
    Does this correspond most closely to response time or throughput?
}{
    \todo
}

\problem{
    f13f0p3
    w15f0p3
    w19f0q19
    w20q2q1
    w20m1q12
}{
    What is a benchmark program?
}{
    Programs that are representative to what we really do with the computers.
}

\problem{
    w15f0p3
    w19f0q20
    w20m1q13
}{
    Do benchmark programs remain valid indefinitely? Why or why not?
}{
    No

    \todo
}

\problem{
    w20q2q3
}{
    We talked about 4 classes of benchmarks.
    What are they?
}{
    \todo
}

\problem{
    w20q2q4
}{
    According to the authros, what is the most reported benchmark for embedded processors.
}{
    \todo
}

\problem{
    w20q2q5
}{
    According to the book, what should be the guiding principle of reporting performance measurement be?
}{
    Reproducibility.
}

\problem{
    w13f0p2
    w15f0p1
    w19f0q15
    w20q2q6
    w20m1q3
}{
    There are two main ways to define performance - what are they?
}{
    Throughput and latency.
}

\problem{
    w20q2q7
}{
    According to the book, a widely held rule of thumb is that in only \blank of the code program spends \blank of it's execution time.
}{
    10\%, 90\%
}

\problem{
    w15f0p1
    w19f0q4
}{
    Is peak performance usually the same as sustained performance.`'

    Is peak performance of a processor a good indicator of what the observed performance will be?
}{
    No, No
}

\problem{
    f13f0p13
    w20q2q8
    w20m1q21
}{
    (CPU time calculation, sampled from w20q2q8)

    \nop
}{
    \begin{align*}
        & 200 \texttt{ inst} \cdot 5 \texttt{ cycle/inst} \cdot 3 \texttt{ ns/cycle} = 3000 \texttt{ns} \\
        & 150 \texttt{ inst} \cdot 4 \texttt{ cycle/inst} \cdot x \texttt{ ns/cycle} = 3000 \texttt{ns} \\
    \end{align*}
    $$ t = 5 \texttt{ ns/cycle}$$
}

\problem{
    f13f0p5
    f15f0p8
    w19f0q35
    w20q2q9
    w20m1q22
}{
    (Hardware upgrade choice, sampled from w19f0q35)
    
    An important program spends 30\% of its time doing memory operations(loads and stores). 
    By redesigning the memory hierarchy you can make the memory operations 80\% faster (take 20\% as long), or you can redesign the hardware to make the rest of the machine 30\% faster (take 70\% as long). 
    Which should you do and why? 
    (You must show your work to get full credit.)
}{
    \todo
}

\problem{
    f13f0p5
}{
    (Change of instruction set)

    \nop
}{
    \todo
}

\problem{
    w20q2q10
    w20m1q24
}{
    (Two architectures that deal with branch differently)

    \nop
}{
    \todo
}

\problem{
    w20m1q8
}{
    Salesmen will often quote the peak performance of their machines.
     What is peak performance, and what other kind of performance should you want to know? 
}{
    \todo
}