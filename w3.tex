\section*{Week 3 (1/20 - 1/26; Appx.A, Quiz 2)}

\problem{
    w19f0q15
}{
    There are two main ways to define performance - what are they?
}{
    Throughput and latency.
}

\problem{
    w19f0q16
}{
    There are two major challenges to obtaining a substantial decrease in response time when using the MIMD approach. 
    What are they?
}{
    \todo
}

\problem{
    w19f0q17
}{
    If I add processors but keep the job size the same, am I measuring strong or weak scaling? 
    Does this correspond most closely to response time or throughput?
}{
    \todo
}

\problem{
    w19f0q4
}{
    Is peak performance usually the same as sustained performance
}{
    \todo
}

\problem{
    w19f0q19
    w20q2q1
}{
    What is a benchmark program?
}{
    \todo
}

\problem{
    w19f0q20
}{
    Do benchmark programs remain valid indefinitely? Why or why not?
}{
    No

    \todo
}

\problem{
    \xxxxxx
    w19f0q21
}{
    Why is it difficult to come up with good benchmarks that will work across all types of parallel processors?
}{
    Because the message-passing machine and shared memory machine works differently.
    They have different algorithms, and different programming approaches.
}

\problem{
    \xxxxxx
    w19f0q35
    w20q2q9
}{
    (Hardware upgrade choice, sampled from w19f0q35)
    
    An important program spends 30\% of its time doing memory operations(loads and stores). 
    By redesigning the memory hierarchy you can make the memory operations 80\% faster (take 20\% as long), or you can redesign the hardware to make the rest of the machine 30\% faster (take 70\% as long). 
    Which should you do and why? 
    (You must show your work to get full credit.)
}{
    \todo
}
