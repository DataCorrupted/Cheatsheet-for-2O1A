\section*{Quiz 3 (Feb 7, 2020)}

\problem{
    f12m1q13
    f13m1q12
    f15m1q12
    w19m1q14
    w20q3q1
}{
    What are the 3 pipeline hazards? Which one can be solved by providing more resources?
}{
    Structural / Data / Control hazards

    Structural hazard can be solved.
}

\problem{
    f05m1q13
    w12m1q15
    w12f0p4
    f14m1q11
    f16m1q11
    f15m1q11
    w20q3q2
}{
    What are the 4 types of data hazards? 
    Which one can be solved by using forwarding?
    Which one is not really a hazard? Why?
}{
    RAW / WAR / WAW

    RAW can be solved by forwarding.

    \todo
}

\problem{
    \xxxxxx
    w20q3q3
}{
    When dealing with control hazards, it is not enough to predict the branch direction, what else must we know?
}{
    Branch target address
}

\problem{
    \xxxxxx
    w20q3q4
}{
    Which is on average more effective, dynamic or static branch prediction?
}{
    Dynamic
}

\problem{
    \xxxxxx
    w20q3q5
}{
    \nop
}{
    a) 48

    b) No, Memory takes 2x time than the other. 
    Split M into 2 equally stages, each taking 24 time-units.
}

\problem{
    \xxxxxx
    w20q3q7
}{
    A simple dynamic branch prediction scheme is to use a table of 2-bit predictors, accessed using bits from the PC. 
    It would seem that the bigger the table, the better the predictor would work, is that true? 
    If not, what is the limit on how many entries the table can have and still give good perfor­mance?
}{
    False, 4096 bits.
}

\problem{
    \xxxxxx
    w20q3q8
}{
    \nop
}{
    \begin{enumerate}
        \item Interrupt on \texttt{i+1} and squash the next two.
        \item Squash \texttt{i+1} to \texttt{i+5}
        \item 5, WB stage cannot go wrong.
    \end{enumerate}
}
