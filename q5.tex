\section*{Quiz 5 (Feb 21, 2020)}

\problem{
    w20q5q1
    \xxxxxx
}{
    Assuming a 5-way multiple issue processor with a 9-stage pipeline and no bubbles, how many instructions are in execution at any time?
}{
    $ 5 \times 9 = 45 $
}

\problem{
    w20q5q2
    \xxxxxx
}{
    As of 2011, how many processors combine full speculation with resolving multiple branches per cycle?
}{
    No processor
}

\problem{
    w20q5q3
    \xxxxxx
}{
    Tomasulo's algorithm attempts to achieve something called Data Flow execution. 
    How does the book define Data Flow execution?
}{
    \pg{184} Operation executes as soon as their operands are available.
}


\problem{
    f04m1q16
    w10m1q10
    w12f0p1
    f13f0p1
    f14m2p2
    f15m2q12
    f15f0p2
    f15f1p1
    f16m2q11
    f16f0p1
    w19m2q10
    w2015q4
}{
    What is the primary difference between superscalar and VLIW processors? 
    Give one advantage and one disadvantage of using each approach. 
    (Compare and contrast VLIW and superscalar with two advantage and one disadvantage.)
    (These have to be different - in other words, if the advantage of superscalar is X, then you can’t say a disadvantage of VLIW is that it can’t do X.)
}{
    \textbf{Superscalar}: hardware does dynamic scheduling  \\
    Pro: better at prediction and disambiguation \& no recompilation on hardware changes \\
    Con: more power consumption \& HW complexity.

    \textbf{VLIW}: complier does static scheduling \\ 
    Pro: simpler hardware, save cost \& power efficient\\
    Con: cannot use runtime information \& \warn{Memory ambiguity}\\
}

\problem{
    w20q5q5
    \xxxxxx
}{
    The book talks about the limits of ILP and mentions that even when using a perfect model there are some limitations. 
    What are the most important limitations that apply even to the perfect model?
}{
    \pg{220}
    \begin{enumerate}
        \item WAW \& WAR through the memory system
        \item Unnecessary dependencies
        \item Overcoming data flow limits
    \end{enumerate}
}

\problem{
    f16m2q5
    f16f0p4
    w19m2q8
    w20q5q6
}{
    Processors have been built that are able to issue 8 instructions at a time. 
    However, these pro­cessors are no longer being built - why not? 
    Why would you choose a 3-issue machine over an 8-issue machine?
}{
    Speculation requires a lot of power, and we don't get 8-issue very often because we don't always find that much parallelism, so it is not power efficient.
    I am wasting all of my energy supporting my 8-issue when I only get 3 or 4.

    OR

    8-issue machine has more hardware and does more speculation, and thus requires lots of power. 
    If we don't have enough parallelism available, it will have many bubbles and waste energy 
    Back then we don't care about that, but now the condition has changed and the energy efficiency is more significant now.

}

\problem{
    f12m2q15
    w20q5q7
}{
    Register renaming is used to avoid name hazards. 
    Is there a technique that can be used to minimize the number of stalls due to true (RAW) hazards? 
    If so, what is it and how does it work?
}{
    Value prediction. 
    Identify when values are being used, and using them in advance when they are needed.
}

\problem{
    w20q5q8
}{
    (Dependencies recognization \& register substitution) \\
    \nop
}{
    a)
    \begin{itemize}
        \item \RAW{1}{4}{F3}  
        \item \WAR{1}{3}{F5}      
        \item \WAR{1}{2}{F7}
        \item \WAR{1}{4}{F7}
        \item \WAR{2}{3}{F5}
        \item \RAW{2}{3}{F7}
        \item \WAW{2}{4}{F7}
        \item \WAR{3}{4}{F7}
    \end{itemize}

    b)
    15, 14, 13, 7, 11, 5, 9, 4

    c)
    \begin{itemize}
        \item \RAW{1}{4}{P7}
        \item \RAW{2}{3}{P6}
    \end{itemize}

}

\problem{
    w20q5q9
}{
    (Fine / Coarse scheduling) \\
    \nop
}{
    a)\\
        A A A   \\
        A A     \\
        B       \\
        A A     \\
        B       \\
        B B     \\

    b)\\
        A A A   \\
        B       \\
        C C     \\
        A A A   \\
        B       \\
        C       \\

    c)\\
        A A B C C \\
        A A A B C \\
        B B C C C \\
        A A B B B \\
}

