\section*{Week 8 (2/24 - 3/1; Chp.4 \& Appx.B, Midterm 2)}

\problem{
    \xxxxxx
}{
    What is the critical advantage of a vector instruction set?
}{
    It allows the software to pass a large amount of parallel work to hardware using only a single short instruction.
}

\problem{
    f15m2q11
    f16m2q10
    w19m2q2
}{
    Briefly outline how a Vector machine works, and what type of parallelism it is exploiting.
}{
    There are several vector registers(VR), fill them up. 
    Single instruction operated on these VRs, such that the single instruction do multiple operations on elements of the VRs. 

    Data level parallelism.
}

\problem{
    w19m2q7
 }{
    A vector processor is a type of SIMD architecture. 
    What makes it different than a "normal" SIMD machine? 
    (Why does it have a separate section in the SIMD chapter in the book?)
}{
    Vector computers processed the vectors one word at a time through pipelined processors 
    (though still based on a single instruction)
    Whereas modern SIMD computers process all elements of the vector simultaneously.
}


\problem{
    f16m2q1
    w19m2q3
    \xxxxxx
}{
    Writes to a cache are inherently slower than reads from a cache - why?
}{
    Read can do check tags and read data in parallel. 
    However, writing is inherently serial/sequential, so we have to check tags before writing the data to the cache. 
}

\problem{
    f15f1p7
    f16m2q16
    w19f0q52
    \xxxxxx
}{
    Assuming a 19-bit address and a 256-byte Direct Mapped cache with a linesize=2, show how an address is partitioned/interpreted by the cache.
}{
    \begin{align*}
        & \texttt{offset} = 1 \texttt{bit }   \\
        & 256/2 = 128 =  2^7                  \\
        & \texttt{entry} = 7 \texttt{bits}    \\
        & \texttt{tag}  = 11\texttt{ bits}
    \end{align*}
}

\problem{
    f15f1p7
    f16m2q17
    w19f0q53
    \xxxxxx
}{
    Assuming a 19-bit address and an 80-byte 10-way Set Associative cache with a linesize=4, show how an address is partitioned/interpreted by the cache.
}{
    \begin{align*}
        & \texttt{offset} = 2 \texttt{bits}   \\
        & 80/(10*4) = 2 = 2^1                 \\
        & \texttt{  set} = 1 \texttt{bit }    \\
        & \texttt{tag} = 16 \texttt{bits}
    \end{align*}
}

\problem{
    f15f1p7
    f16m2q18
    w19f0q54
    \xxxxxx
}{
    Assuming a 19-bit address and a 328-byte Fully Associative cache with a linesize=8, show how an address is partitioned/interpreted by the cache.
}{
    \begin{align*}
        & \texttt{offset} =  3 \texttt{bits} \\
        & \texttt{tag   } = 16 \texttt{bits}        
    \end{align*}
}

\problem{
    f15f1p7
    f16m2q19
    w19f0q55
    \xxxxxx
}{
    Given a 1 Megabyte physical memory, a 22-bit Virtual address, and a page size of 1K bytes, write down the number of entries in the Page Table, and the width of each entry.
}{
    \begin{align*}
        & 1M = 2^{20}; 1K = 2^{10} \\
        & (2^{20})/(2^{10}) = 2^{10} \rightarrow \texttt{10-bit wide }
    \end{align*}
    PM: $ 10 | 10 $ \\
    VM: $ 12 | 10 $ \\
    $2^{12}$ \texttt{entries}        
}

\problem{
    f15f1p7
    f16m2q20
    w19f0q56
    \xxxxxx
}{
    Given a 1 Megabyte physical memory, a 34-bit Virtual address, and a page size of 2K bytes, write down the number of entries in the Page Table, and the width of each entry. Is there a problem with this configuration? If so, what is the problem?
}{
    \begin{align*}
        & 1M = 2^{20}; 2K = 2^{11} \\
        & (2^{20})/(2^{11}) = 2^9 \rightarrow \texttt{ 9-bit wide }
    \end{align*}
    PM: $  9 | 11 $ \\
    VM: $ 23 | 11 $ \\
    $2^{23}$ \texttt{entries}         \\
    Problem: It is too large to store in 1M physical memory. We can use two page tables. One is base, another one is secondary.
}