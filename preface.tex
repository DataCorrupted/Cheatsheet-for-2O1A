\newSection{Preface \& Usage guide}

\textbf{\textit{``Your answer should convey understanding.''}}

This is Farren's words when teaching ECS201A: Computer Architecture.
At first glance it's not a big deal, yet, this is how understanding looks like: 
``The first midterm I wrote a lot, and he said it looks like I am guessing the answer.
The second I made it shorter with few words, then he said they don't convey understanding and every wrong answer in this class looked similar, implying I am cheating.
Now I am going to use his exact words.''

\textbf{\textit{``You are graduate students, you should figure that out.''}}

This is Farren's words again. 
Yes, and this \name is how we figured that out.
This is how it's gonna work, in this project we are going to write down every word he said and all you need to do is remember them.
To be short, this repo is what ``understanding'' looks like, once and for all.
Ideally, this should be a repo for all problems in ECS201A.

This preamble section should serve as a README section to help you use this project to prepare for EC201 Computer Architecture's quizes, midterms and final.
Should you have anything to discuss or add, send us an issue or email contributors listed in Appx.3.

\newSubSection{Usage}

Type in \lstinline{make} to compile to a pdf file.
If the pdf is not opened automatically after compilation you may open \texttt{Makefile} and set up a new pdf viewer.
You can turn \lstinline{showAns} in \texttt{main.tex:61} to \lstinline{false} to hide answers so you can do practises.

\newSubSection{Selection pool}

As we hope we can go over and filter all the exams we have in hand, it is not possible with too few of us.
However, \href{https://en.wikipedia.org/wiki/Law_of_large_numbers}{\textit{Law of large numbers}} tells us as long as we go over enough numbers we would get pretty close to the mean.
That's the reason we set up this selection pool, i.e. we only look for questions asked in this pool. 
We would record if the same question appeared outside the pool, just we won't be actively looking for new problems outside the pool.

Please use \lstinline|\st{<content>}| to strikethrough \st{contents} in the pool or add your names after it so that we know that this part has been done/been working on.

The pool consists of the following:

\def\width{0.24}
\begin{minipage}[t]{\width\textwidth}
    \subsubsection*{Winter 20}
    \begin{itemize}
        \item \st{\texttt{w20q1}}
        \item \st{\texttt{w20q2}}
        \item \st{\texttt{w20m1}}
        \item \st{\texttt{w20q3}}
        \item \st{\texttt{w20q4}}
        \item \st{\texttt{w20q5}}
        \item \st{\texttt{w20m2}}
        \item \st{\texttt{w20q6}}
        \item \texttt{w20m3}
    \end{itemize}
\end{minipage}
\begin{minipage}[t]{\width\textwidth}
    \subsubsection*{Midterm1s}
    \begin{itemize}
        \item \texttt{f12m1}
        \item \texttt{w12m1}
        \item \texttt{f13m1}
        \item \texttt{f14m1}
        \item \texttt{w19m1}
    \end{itemize}    
\end{minipage}
\begin{minipage}[t]{\width\textwidth}
    \subsubsection*{Midterm2s}
    \begin{itemize}
        \item \texttt{f10m2}
        \item \texttt{f12m2}
        \item \st{\texttt{f14m2}}
        \item \st{\texttt{f15m2}}
        \item \st{\texttt{f16m2}}
        \item \st{\texttt{w19m2}}
    \end{itemize}    
\end{minipage}
\begin{minipage}[t]{\width\textwidth}
    \subsubsection*{Finals}
    \begin{itemize}
        \item \st{\texttt{f13f0}}
        \item \st{\texttt{f15f0}}
        \item \texttt{f16f0}
        \item \st{\texttt{w19f0}}
    \end{itemize}
\end{minipage}