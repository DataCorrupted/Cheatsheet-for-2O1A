\section*{Week 4 (1/27 - 2/2; Appx.C, Midterm 1)}

\problem{
    w20m1q5
}{
    If a data word is aligned(where a word is 32 bits), what does that mean?
    (What can be said about the address of the begining of the data item?)
}{
    The beginning address always ends with 2 zeros.
}

\problem{
    w13f0p1
    w15f0p1
    w19f0q51
    w20m1q6
}{
    When we talk about the number of operands in an instruction 
    (a 1-operand or a 2-operand instruction, for example), what do we mean?
}{
    The number of operands that are \textbf{explicitly} used in instruction.
}

\problem{
    w19m1q14
}{
    What are the two primary goals of a compiler(in order)?
}{
    \begin{itemize}
        \item Make programs correct when compiled
        \item Make them fast
    \end{itemize}
}


\problem{
    w20m1q19
}{
    Machines today use registers - often as many as they can. 
    Give 2 advantages and 2 disadvantages
}{
    \begin{enumerate}
        \item Pros
        \begin{itemize}
            \item Fast
            \item Deterministic
        \end{itemize} 
        \item Cons
        \begin{itemize}
            \item Fixed-size
            \item Hot
            \item Inter-procedural hardship(Caller saved \& callee saved)
        \end{itemize}
    \end{enumerate}
}

\problem{
    w20m1q20
}{
    A standard compiler optimization step is to do register allocation. 
    However, there are times when register allocation is difficult to do while maintaining program correctness - explain why.
}{
    \todo
}

\problem{
    f13f0p5
    f15f0p8
    w20m1q23
}{
    (Change of instruction set)
    
    You are responsible for designing a new embedded processor, and you have been told you must use a fixed 18-bit instruction size. To make this work you have decided to use a 2-operand instruc­tion format, 32 registers, and support 256 instructions. Your boss just came in and said things have changed, and you now can only use 12 bits - how would you change your instruction format? 
    Be sure to explain why you are making the changes (what problem are you solving?) 
}{
    \todo
}

\problem{
    w20m1q24
}{
    Suppose we are considering two alternatives for our conditional branch instructions, as follows:
    \begin{itemize}
        \item M1 - A condition code is set by a compare instruction and followed by a branch that tests the condition 
        \item M2 - A compare is included in the branch. 
    \end{itemize}
    On both machines, the conditional branch instruction takes 2 cycles, and all other instructions take 1 cycle. 
    On Ml, 20\% of all instructions executed are conditional branches. 
    Because M2 has the com­pare included in the branch, the clock cycle is 1.10 times longer than M1. 
}{
    \todo
}