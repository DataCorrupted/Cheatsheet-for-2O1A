\section*{Contribution \& usage guide}
This is \name. 
This preamble section should serve as a README section to help you contribute to this project and prepare for exams.
Ideally, this should be a repo for all problems in ECS201A: Computer Architecture.
Should you have anything to discuss or add, send us an email or issue.

\subsection*{Contribution}

\subsubsection*{How to add a problem}
In this file, all questions are recorded and answered.
A question is composed of 3 parts: \textit{Appearance}, \textit{Body}, and \textit{Ans}.

\textit{Appearance} includes the quarter, exam id and question id of this question so we can refer to where exactly it is.
The first two denotes the season and year, for example, \texttt{f15} denotes fall 15.
The middle two denotes the exam it appeared on, \texttt{f0} refers to final, midterms and quizzes use letter \texttt{m} and \texttt{q}, attached with the number showing which midter/quiz it is.
The final component starts with \texttt{q} and a question id after it, if there is no label on question id, use page number starting with \texttt{p}
For example, \texttt{f17m1q4} means the 4th question in midterm 1 in fall 2017. 
If there is only one midterm in that quarter, use \texttt{m0}.
Each \textit{appearance} should take up a line to avoid merge conflict. 
It is suggested that you put \textit{appearance} in the order of time.
If you don't have time to find all \textit{appearance}, use command \lstinline{\xxxxxx} instead.

\texttt{Body} refers to the question itself. 
Different exams may have altered questions, please try to put a superset of all similar ones.
If the body is too long or involves code/drawing, you may skip it by put command \lstinline{\nop} here.
Then it would be crucial that you put correct \textit{Appearance} so we can find it.

Please provide most accurate \texttt{Ans} possible.

Please use command \texttt{problem} to add a problem.
For code hygiene, each sentence should only take up one line to avoid merge conflict.
Please use the following format:
\begin{lstlisting}
\problem{
    <Appearance1>
    <Appearance2>
    <Appearance3>
} {
    <Body>
} {
    <Ans>
}
\end{lstlisting}

If you put 
\begin{lstlisting}
\problem{
    f12m1q1
    w12m1q1
    f12f0p1
    w12f0p1
    f14m1q3
    f13m1q4
    f14f0p1
    f15f0p1
}{
    As minimum feature sizes decrease, 
    a) what happens to power density?
    b) what happens to wire resistance, and why?
}{
    Density goes up and resistance goes down.
}
\end{lstlisting}
you get:

\problem{
    f12m1q1
    w12m1q1
    f12f0p1
    w12f0p1
    f14m1q3
    f13m1q4
    f14f0p1
    f15f0p1
}{
    As minimum feature sizes decrease, a) what happens to power density? b) what happens to wire resistance, and why?
}{
    Density goes up and resistance goes down.
}

\subsubsection*{Where to put a problem}

We want to arrange all the problems in the order of our classes.
So we set up different files, including \texttt{q1.tex}, \texttt{q2.tex}, \texttt{m1.tex}, etc.
All questions before quiz 1(or all questions related to the ones in \texttt{q1}) should be put in \texttt{q1}.
Similarly, put everything between quiz4 and midterm2 in \texttt{m2.tex}.

\subsubsection*{Pro tip on how to find all \textit{Appearance}}

Farrens (unwisely) put all his pdf under one publically available folder, making all files easy to craw.
Use google search to target on his website using \texttt{site} keyword, search on important keyword and you are good to go.

For example, to get all \textit{Appearance} on the problem we mentioned above, google search \texttt{minimum feature size decreases site:http://american.cs.ucdavis.edu/academic/ecs201a/postscript/}


\subsection*{Usage}

Type in \lstinline{make} to compile to a pdf file. 
You can turn \texttt{showAns} in \texttt{main.tex:36} to \texttt{false} to hide answers.
Should you have any questions, feel free to contact us.

\subsection*{Selection pool}

As we hope we can go over and filter all the exams we have in hand, it is not possible with too few of us.
However, probability tells us as long as we go over enough numbers we would approach real mean.
That's the reason we set up this selection pool, i.e. we only look for questions asked in this pool. 
We would record if the same question appeared outside the pool, just we won't be actively looking for new problems outside the pool.

Please use \lstinline|\st{<content>}| to strikethrough \st{contents} in the pool or add your names after it so that we know that this part has been done.

The pool consists of the following:

\begin{minipage}[t]{0.24\textwidth}
    \subsubsection*{Winter 20}
    \begin{itemize}
        \item \texttt{w20q1}
        \item \texttt{w20q2}
        \item \texttt{w20m1}
        \item \st{\texttt{w20q3}}
        \item \st{\texttt{w20q4}}
        \item \st{\texttt{w20q5}}
        \item \texttt{w20m2}
    \end{itemize}
\end{minipage}
\begin{minipage}[t]{0.24\textwidth}
    \subsubsection*{Midterm1s}
    \begin{itemize}
        \item \texttt{f12m1}
        \item \texttt{w12m1}
        \item \texttt{f13m1}
        \item \texttt{f14m1}
        \item \texttt{w19m1}
    \end{itemize}    
\end{minipage}
\begin{minipage}[t]{0.24\textwidth}
    \subsubsection*{Midterm2s}
    \begin{itemize}
        \item \texttt{f10m2}
        \item \texttt{f12m2}
        \item \st{\texttt{f14m2}(Kent)}
        \item \st{\texttt{f15m2}(Peter)}
        \item \texttt{f16m2}(Xiaoyu)
    \end{itemize}    
\end{minipage}
\begin{minipage}[t]{0.24\textwidth}
    \subsubsection*{Finals}
    \begin{itemize}
        \item \texttt{f13f0}
        \item \texttt{f15f0}
        \item \texttt{f16f0}
    \end{itemize}
\end{minipage}